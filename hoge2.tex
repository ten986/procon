\hypertarget{ux898bux51faux3057}{%
\section{見出し}\label{ux898bux51faux3057}}

\hypertarget{ux898bux51faux30572}{%
\subsection{見出し2}\label{ux898bux51faux30572}}

いい感じの文章を書いてレポートの本文をうめていく。改行したいときは見難いがこのように→
スペースを 2 つつければ改行できる

1 行空行を空けると{[}\^{}1{]}別のパラグラフになる。

\hypertarget{tbl:table}{}
\begin{longtable}[]{@{}rc@{}}
\caption{\label{tbl:table}すごい表}\tabularnewline
\toprule
i & サイコロの目\tabularnewline
\midrule
\endfirsthead
\toprule
i & サイコロの目\tabularnewline
\midrule
\endhead
1 & 3\tabularnewline
2 & 2\tabularnewline
3 & 6\tabularnewline
4 & 5\tabularnewline
5 & 1\tabularnewline
6 & 4\tabularnewline
7 & 2\tabularnewline
8 & 6\tabularnewline
\bottomrule
\end{longtable}

\hypertarget{tbl:table-long}{}
\begin{longtable}[]{@{}crr@{}}
\caption{\label{tbl:table-long}へんな表}\tabularnewline
\toprule
\begin{minipage}[b]{0.18\columnwidth}\centering
サンプル\strut
\end{minipage} & \begin{minipage}[b]{0.37\columnwidth}\raggedleft
見た目 / \({\rm m^2 \cdot kg \cdot s^{-3} \cdot A^{-1}}\)\strut
\end{minipage} & \begin{minipage}[b]{0.37\columnwidth}\raggedleft
雰囲気 / \({\rm m^{-2} \cdot kg^{-1} \cdot s^{4} \cdot A^{2}}\)\strut
\end{minipage}\tabularnewline
\midrule
\endfirsthead
\toprule
\begin{minipage}[b]{0.18\columnwidth}\centering
サンプル\strut
\end{minipage} & \begin{minipage}[b]{0.37\columnwidth}\raggedleft
見た目 / \({\rm m^2 \cdot kg \cdot s^{-3} \cdot A^{-1}}\)\strut
\end{minipage} & \begin{minipage}[b]{0.37\columnwidth}\raggedleft
雰囲気 / \({\rm m^{-2} \cdot kg^{-1} \cdot s^{4} \cdot A^{2}}\)\strut
\end{minipage}\tabularnewline
\midrule
\endhead
\begin{minipage}[t]{0.18\columnwidth}\centering
ポカリスエット\strut
\end{minipage} & \begin{minipage}[t]{0.37\columnwidth}\raggedleft
2\strut
\end{minipage} & \begin{minipage}[t]{0.37\columnwidth}\raggedleft
40\strut
\end{minipage}\tabularnewline
\begin{minipage}[t]{0.18\columnwidth}\centering
アクエリアス\strut
\end{minipage} & \begin{minipage}[t]{0.37\columnwidth}\raggedleft
2\strut
\end{minipage} & \begin{minipage}[t]{0.37\columnwidth}\raggedleft
21\strut
\end{minipage}\tabularnewline
\begin{minipage}[t]{0.18\columnwidth}\centering
ダカラ\strut
\end{minipage} & \begin{minipage}[t]{0.37\columnwidth}\raggedleft
3\strut
\end{minipage} & \begin{minipage}[t]{0.37\columnwidth}\raggedleft
8\strut
\end{minipage}\tabularnewline
\bottomrule
\end{longtable}

この辺で改ページ \clearpage

\hypertarget{ux6ce8ux610f}{%
\subsection{注意}\label{ux6ce8ux610f}}

\begin{enumerate}
\def\labelenumi{\arabic{enumi}.}


\item
  fig.~\ref{fig:sugoi} は普通の図
\item
  fig.~\ref{fig:sugoi-large} のような大きい図を貼るときは height=70mm
  などとして適当に調整する
\item
  tbl.~\ref{tbl:table} のように表にラベルをつけるときは
  \texttt{:タイトル\ \{\#tbl:tabl\}} のように \texttt{\{\}}
  の前にスペースが必要
\item
  tbl.~\ref{tbl:table-long} のように表のヘッダーが長いときは
  \texttt{:-\/-\/-\/-\/-\/-\/-\/-\/-\/-\/-\/-\/-\/-\/-\/-\/-\/-\/-\/-\/-\/-\/-:}
  みたいに 2 行目の \texttt{-} を増やせば表の横幅が広くなったりする
\end{enumerate}

\hypertarget{ux3044ux308dux3044ux308dux306aux5f0f}{%
\subsection{いろいろな式}\label{ux3044ux308dux3044ux308dux306aux5f0f}}

\hypertarget{ux30a4ux30f3ux30e9ux30a4ux30f3ux306eux5f0f}{%
\subsubsection{インラインの式}\label{ux30a4ux30f3ux30e9ux30a4ux30f3ux306eux5f0f}}

本文中に唐突に \(E = mc^2\) 埋め込む。

インラインの式中に
\(a_n = \frac{1}{\pi} \int_{0}^{2\pi} f(x) \cos nx dx\)
のような分数やインテグラルが入るとちょっと見にくくなるので、\(\displaystyle b_n = \frac{1}{\pi} \int_{0}^{2\pi} f(x) \sin nx dx\)
とすると見やすくなる。

\hypertarget{ux3088ux304fux3042ux308bux5f0f}{%
\subsubsection{よくある式}\label{ux3088ux304fux3042ux308bux5f0f}}

\begin{equation}f(x) = \frac{a_0}{2} + \sum_{n = 1}^{\infty} a_n \cos nx + b_n \sin nx\label{eq:fourier}\end{equation}

こうやって eq.~\ref{eq:fourier} 参照する。

\hypertarget{ux3088ux304fux3064ux304bux3046ux30aeux30eaux30b7ux30e3ux6587ux5b57ux305fux3061}{%
\subsubsection{よくつかうギリシャ文字たち}\label{ux3088ux304fux3064ux304bux3046ux30aeux30eaux30b7ux30e3ux6587ux5b57ux305fux3061}}

\(\alpha, \beta, \gamma, \delta, \Delta, \varepsilon, \theta, \lambda, \mu, \nu, \pi, \rho, \sigma, \Sigma, \tau, \phi, \omega\)

\[\frac{\partial f}{\partial y} \frac{d f}{d x}\]

\hypertarget{ux30bdux30fcux30b9ux30b3ux30fcux30c9}{%
\subsection{ソースコード}\label{ux30bdux30fcux30b9ux30b3ux30fcux30c9}}

'''\{\#lst:awesome-code .javascript .numberLines startFrom=``10''
caption=``すごいコード''\} var gulp = require(``gulp''); var browserify
= require(``browserify''); var source =
require(``vinyl-source-stream'');

gulp.task(``es6'', function() \{ return browserify(``./src/app.js'')
.transform(``babelify'') .bundle() .pipe(source(``bundle.js''))
.pipe(gulp.dest(``dist'')); \}); '''

↑↓ ここは ' ではなく本当は `

''`\$ cd \$ mkdir -p projectX \$ pbpaste \textgreater{}
projectX/gulpfile.js'''

lst.~\ref{lst:awesome-code}
のようにするとシンタックスハイライトもできるし行番号もつけれたりするが、長いコードだとページ上の配置や改ページが微妙になることがあってあまり使い勝手がよくなかったりする。
どうせ白黒で印刷するなら下の例のようにしたほうがいい。

\hypertarget{ux898bux51faux30573}{%
\subsubsection{見出し3}\label{ux898bux51faux30573}}

\hypertarget{ux898bux51faux30574}{%
\paragraph{見出し4}\label{ux898bux51faux30574}}

\hypertarget{ux898bux51faux30575}{%
\subparagraph{見出し5}\label{ux898bux51faux30575}}

\hypertarget{ux53c2ux8003ux6587ux732e}{%
\section*{参考文献}\label{ux53c2ux8003ux6587ux732e}}
\addcontentsline{toc}{section}{参考文献}

\texttt{\{-\}}
をつけるとこのセクションには見出しに通し番号がつかなくなる

\begin{itemize}


\item
  箇条書き
\item
  箇条書き

  \begin{itemize}
  

  \item
    ネスト
  \item
    ネスト
  \item
    ネスト
  \end{itemize}
\end{itemize}
